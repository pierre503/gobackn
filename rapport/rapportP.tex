\documentclass[a4paper,10pt]{article}
\usepackage[utf8]{inputenc}

%opening
\title{Projet : Implémentation d'un protocole de type Go-Back-N avec contrôle de la congestion}
\author{Maazouz Mehdi, Zielinski Pierre}

\begin{document}

\maketitle
\textbf{Annee Academique 2016-2017}\\
\tableofcontents
\newpage

\section{Introduction}
Dans le cadre du cours de Reseau, nous avons eu comme objectif d'implémenter un protocole de type Go-Back-N
avec un contrôle de la congestion de type Reno au sein d'un simulateur fourni par les professeurs. Le tout était 
à effectuer en Java.

\section{Lancement}
Lorsque vous voulez lancer le projet, vous pouvez définir plusieurs variables qui influeront sur le programme.\\
Tout d'abord, en première position, vous pouvez définir le nombre de paquets que vous voulez envoyer en première variable.\\
\\
En deuxième position vous pouvez aussi définir la position de depart du treshold.\\
\\
De plus, en troisième position, si vous voulez tester plus aisément la perte de paquets, vous pouvez définir le pourcentage de perte des paquets sur le lien.
( il doit être compris en 0 et 100 ).\\
\\
Enfin, dernière position, vous pouvez aussi choisir le pourcentage de perte de ACK( il doit être compris en 0 et 100 ).\\
\\
Pour plus de facilité lors de l'Implémentation, à partir du moment où vous souhaitez changer une de ces variables, veuillez indiquer 0 pour toutes les variables précédentes (exemple si vous voulez juste changer le pourcentage de perte de paquets vous devez lancer le programme suivi de 0,0,20).\\
\\
En suivant ces instructions vous pourrez lancer le programme par la classe "LauncherGoBackN" facilement et celon vos critère.

\section{Utilisation du SenderProtocol et du GoBackNProtocol dans d'autre simulation}
Le SenderProtocol et le GoBackNProtocol peuvent être utilisés dans d'autres simulations. Ils doivent cependant être lancé ensemble car ils fonctionnent ensemble. Une fois, le SenderProtocol et le GoBackNProtocol créés, et rajoutés dans un ip listener, vous pouvez lancer la communication entre les deux en lançant la méthode "launch" du SenderProtocol. Après la méthode launch lancée vous pouvez rajouter des messages au SenderProtocol (qui s'occupera de l'envoi des messages) de deux façons. Soit vous pouvez envoyer  une ArrayList de messages (sous forme d'entier de 32 bits) avec la méthode "addMessageTosend", soit rajouter directement une ArrayList de "PayloadMessage" mais ceux-ci doivent alors déjà avoir leur bon numéro de séquence ou alors ils vont créer des gestions de timeout à l'infini. Une fois tout les messages envoyer à l'instance de SenderProtocol vous devez lui signaler que vous avez fini en laçant la méthode "end" qui signalera que vous ne voulez plus envoyer de messages et que l'application peut s'arrêter une fois qu'elle a envoyer tout les messages.\\

\section{Lecture des données d'une simulation}
Durant une simulation deux sortes de données vont être créée. Il y aura les données transférées sur le fichier "log.txt" et les données affichées dans le terminal. Les données du fichier texte permettront si elles sont utilisées dans "gnuplot" de voir l'évolution de la taille de la fenêtre d'envoi par rapport au temps et ainsi d'observer visuellement le slow start , l'additive increase, les 3 ACK dupliqués et les time out. L'affichage du terminal montrera des informations plus précise comme la valeur actuel du timer, la taille de la fenêtre d'envoi et la position du curseur se trouvant à l'intérieur, les messages envoyés , les acks envoyés, ... Ces informations sont plus précise et permettent de mieux observer l'évolution du programme. De plus la gestion de 3ACK dupliqué ou de time out y est signalé par des "/!$\backslash$".\\

\section{Choix dans l'implémentation}
Au commencement nous avons créé un programme simple pour envoyer des messages avec des numéros de séquence et qu'une fois ceux-ci reçu par le receveur qu'il confirme la réception avec un ACK. Nous avons donc choisi que nos paquets d'envoi contiendraient un numéro de séquence en binaire, d'une taille de 32 bits suivi du message. Ce qui nous as permis de facilement différencier le numéro de séquence du message.\\
Une fois les paquets bien envoyé et les ACK bien reçus nous sommes passé à la gestion des ACK perdu. Cette partie c'est faites assez rapidement car elle avait déjà été presque entièrement gérée.\\
Nous sommes passez après à la gestion des 3ACK dupliqué ainsi qu'à la gestion du time out avec un timer. Cette partie nous as poussé à gérer la perte de paquets à l'envoi car nous ne pouvions pas observer les gestions des 3ACK dupliqués ou des time out sinon. Nous avons même amélioré la gestion des 3 ACK, car ceux-ci ne peuvent lancer des exceptions qu'une seul fois, c'est-à-dire que si le timer est trop long et qu'il ne se déclenche pas, et qu'il y a 6 ACK dupliqués, nous n'activons qu'une fois la gestion des 3 ACK dupliqués (même si normalement un time out doit logiquement se déclencher avant).\\
Une fois tout ces fonctionnalités implémentées nous avons permis à l'utilisateur de définir des variables du programmes pour pouvoir faire des observations plus proche de celles qu'il souhaiterait.\\
Pour terminer nous avons permis à l'utilisateur de rajouter des éléments à l'envoi, quand il le voulais, durant l'exécution du programmes, et de choisir quand il avait fini d'envoyer ses informations.\\



\section{Problèmes rencontrés}
Tout au long du développement, plusieurs problèmes ont été rencontrés.\\
Afin de déterminer le temps de RTT d'un paquet, nous avions décidé d'utiliser un paquet test composé d'un numéro de séquence -1.
Cependant, due à notre implémentation, le numéro de séquence est transformé en binaire. Ce qui a causé quelques problèmes étant donné
que l'entier était négatif. Nous avons donc décidé que le numéro de séquence 0 serait notre paquet test.\\
\\
Suite à une discussion avec les professeurs, nous avons choisis d'exclure la probabilité de perte sur le paquet test.
Ce qui permet de recevoir un ACK(0) sans aucune ambiguïté.
\\
Suite à des soucis de compréhension du fonctionnement du slow start et de l'additive increase, nous avons du réécrire la partie du code s'y rapportant vu qu'elle n'était pas correct.
\\
Nous avons avons aussi rencontré des problèmes au niveau du timer car nous créions un nouveau Scheduler au lieu d'utiliser celui de l'host.
\\
Nous avons aussi rencontré un problème qui  a été vite résolu au niveau de la gestion de fin d'envoi car à la fin du programmes nous pouvions recevoir plusieurs ACK dupliqué si il y avait eu un time out ou 3 ACK dupliqué et nous lancions une erreur alors que tout les messages étaient envoyé.

\section{Conclusion}
Plusieurs objectifs étaient à atteindre lors de la réalisation de ce projet. Tout d'abord, nous avons dû implémenter le 
protocole de type Go-Back-N, nous avons également implémenter le ``pipelining'' permettant l'envois de plusieurs paquets.
De plus , nous avons également implémenté la congestion comme demandé par les professeurs.\\

\end{document}

